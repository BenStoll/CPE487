\section{Description}
    \begin{itemize}
        \item The project will be a recreation of the hit game Tetris using the Nexys A7 board. The game will be a challenge to port from the internet, however we intend to put our own spin on it by including a few variations.
        \item The first variation will include music that we created and shall change with the inputs provided on the board. This is not decided upon just yet and may be that pressing buttons or flipping switches change the music being played.
        \item The second variation is that the next piece will be displayed on the right hand side of the LED display on the Nexys board or we could see about using an LED Pmod to display the piece. TBD
        \item The third addition is that the game will be controlled using a hex calc numpad pmod.
        \item We plan to achieve a playable model of tetris using the Nexys board.
    \end{itemize}
\section{Requirements}
    \begin{itemize}
        \item We need to acquire the following peripheral modules:
            \begin{itemize}
                \item Monitor
                \item VGA cable
                \item Speaker
                \item Hexcalc numpad
            \end{itemize}
        \item We need to select and implement a music playlist for the audio portion
        \item Visual Game Aspects
        \begin{itemize}
            \item We need the game to create pieces in random orders and then the pieces will need to respond to inputs to change position and orientation. 
            \item Additionally, once the row is filled, the row should be deleted with a sound to indicate a row removed and then the pieces should be shifted down to the bottom.
            \item Lastly, the speed of the pieces should be controllable so the switches on the board indicate difficulty level (AKA speed of the pieces).
            \item The underneath portion is Prof. Muresan's suggestions and warnings for out project:
            \item Some things to think about - We will need to simply just try to get the blocks to fall and then disappear. After that you can have the blocks start to stack and then disappear on a button press. Afterwards we will need to add more of the game in. 
            \item Additionally, we could have a keyboard for the controls nstead of the hex counter via a USB port on the FPGA but this is more of a difficulty. Another difficulty would be the music. My idea for the musc would be to have the pitch change up and down with the L/R and U/D of the piece but tbd. The score would be easy to display however scoring would need to be laid out. Lastly, the block storage would also be pretty complicated to program and may or may not get done. 
            \item Also, we need to add this and the project change log to the Final Project Diagram. Basically standardization of this document to fit within the template.
        \end{itemize}
    \end{itemize}
\section{Block Diagram}
    \includegraphics[width=100mm]{blockDiagram.png}
\section{Risks}
    \begin{itemize}
        \item Copyright infringement and being sued :)
        \item Burning boards/Pmods
    \end{itemize}
\section{Testing}
    \begin{itemize}
        \item We will use test-benches to test the music being played and which pieces will be displayed
    \end{itemize}
