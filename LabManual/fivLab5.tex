\chapter{Lab 5: DAC Siren} \\
\index{Lab 5}
\index{introduction}
\label{Introduction}


\section{Lessons Learned
\label{Section::Lessons Learned}}
In this lab we learned how to program the FPGA on the Nexys A7-100T board to generate a wailing audio siren using a 24-bit Dac called a Pmod.  \\
We:
\begin{itemize}
    \item Practiced coding a Nexys board
    \item Learned to output audio from a Nexys board so a Pmod would play audio
    \item Learned to change upper and lower tone limits
    \item Learned to changed the wailing speed
    \item Learned to add a second wail instance in the right audio channel
 \end{itemize}

 \section{Schematics and Block Diagrams}
 There are wave graphs we can screenshot and add from GitHub\\
 Add the i2s, i2s2, and wave pictures\\
 
 \section{VHDL Architecture}
 The big new idea we learned from this lab is implementing audio when using the Nexys board.
 
 \begin{verbatim}
     siren.xdc

create_clock -name clk_50MHz -period 20.00 [get_ports clk_50MHz] 

set_property -dict { PACKAGE_PIN E3 IOSTANDARD LVCMOS33 } [get_ports clk_50MHz]

set_property -dict { PACKAGE_PIN D18 IOSTANDARD LVCMOS33 } [get_ports { dac_LRCK }]; #IO_L21N_T3_DQS_A18_15 Sch=ja[2]

set_property -dict { PACKAGE_PIN E18 IOSTANDARD LVCMOS33 } [get_ports { dac_SCLK }]; #IO_L21P_T3_DQS_15 Sch=ja[3]

set_property -dict { PACKAGE_PIN G17 IOSTANDARD LVCMOS33 } [get_ports { dac_SDIN }]; #IO_L18N_T2_A23_15 Sch=ja[4]

set_property -dict { PACKAGE_PIN C17 IOSTANDARD LVCMOS33 } [get_ports { dac_MCLK }]; #IO_L20N_T3_A19_15 Sch=ja[1]


set_property -dict { PACKAGE_PIN M18   IOSTANDARD LVCMOS33 } [get_ports { BTN0 }]; #IO_L4N_T0_D05_14 Sch=btnu
set_property -dict { PACKAGE_PIN J15   IOSTANDARD LVCMOS33 } [get_ports { SW0 }]; #IO_L24N_T3_RS0_15 Sch=sw[0]
set_property -dict { PACKAGE_PIN L16   IOSTANDARD LVCMOS33 } [get_ports { SW1 }]; #IO_L3N_T0_DQS_EMCCLK_14 Sch=sw[1]
set_property -dict { PACKAGE_PIN M13   IOSTANDARD LVCMOS33 } [get_ports { SW2 }]; #IO_L6N_T0_D08_VREF_14 Sch=sw[2]
set_property -dict { PACKAGE_PIN R15   IOSTANDARD LVCMOS33 } [get_ports { SW3 }]; #IO_L13N_T2_MRCC_14 Sch=sw[3]
set_property -dict { PACKAGE_PIN R17   IOSTANDARD LVCMOS33 } [get_ports { SW4 }]; #IO_L12N_T1_MRCC_14 Sch=sw[4]
set_property -dict { PACKAGE_PIN T18   IOSTANDARD LVCMOS33 } [get_ports { SW5 }]; #IO_L7N_T1_D10_14 Sch=sw[5]
set_property -dict { PACKAGE_PIN U18   IOSTANDARD LVCMOS33 } [get_ports { SW6 }]; #IO_L17N_T2_A13_D29_14 Sch=sw[6]
set_property -dict { PACKAGE_PIN R13   IOSTANDARD LVCMOS33 } [get_ports { SW7 }]; #IO_L5N_T0_D07_14 Sch=sw[7]

 \end{verbatim}
 
 \section{VHDL Models}
 
 \begin{itemize}
     \item Data Flow
     \begin{itemize}
         \item dac\_if\\
         Input: 16-bit parallel stereo data\\
         Output: serial format conversion through digital to analog converter\\
         \item tone\\
         Input: None\\
         Output: signed triangular wave (a tone)  at sampling rate 48.8 KHz
         \item wail\\
         Input: wspeed and wclk - these determine pitch change\\
         Output: different pitch tone\\
         \item siren\\
         Input: wail audio data, dac\_if converted data\\
         Output: audio data to DAC siren
     \end{itemize}
     \item Structural\\
     - Design Sources\\
         \begin{itemize}
             \item siren.vhd
             \begin{itemize}
                 \item dac\_if.vhd
                 \item w1: wail.vhd
                 \begin{itemize}
                     \item tone.vhd
                 \end{itemize}
                 \item w2: wail.vhd
                 \begin{itemize}
                     \item tone.vhd
                 \end{itemize}
             \end{itemize}
         \end{itemize}
         - Constraints\\
         \begin{itemize}
             \item constrs\_1
             \begin{itemize}
                 \item siren.xdc
             \end{itemize}
         \end{itemize}
         - Simulation Source\\
         -- sim\_1\\
         \begin{itemize}
             \item siren.vhd
             \begin{itemize}
                 \item dac\_if.vhd
                 \item w1: wail.vhd
                 \begin{itemize}
                     \item tone.vhd
                 \end{itemize}
                 \item w2: wail.vhd
                 \begin{itemize}
                     \item tone.vhd
                 \end{itemize}
             \end{itemize}
         \end{itemize}
         - Utility Sources\\
         \begin{itemize}
             \item utils\_1
             \begin{itemize}
                 \item Design Checkpoint
                 \begin{itemize}
                     \item siren.dcp
                 \end{itemize}
             \end{itemize}
         \end{itemize}
     \item Behavioral\\
     \begin{itemize}
        \item BTNU : creates square wave instead of triangle wave
        \item SW0-SW7 : set wailing speed
        \item Different audio channels : high and low tone limits
        \item siren : outputs audio
    \end{itemize}
 \end{itemize}
 \section{VHDL Component Reuse}
 This section reuses the BTNU button - it creates a square wave in this lab.
 \section{VHDL Digital Circuits}
    \includegraphics[width=100mm]{lab5/dacsirenimgthing.png}
    \emph{Figure 5.2: Image of the Digital Architecture and Circuit}
 \section{State Machines}
 System outputs a triangle wave at a set wailing speed until a button is pressed to change the wave, a switch is flipped to change the speed, or the audio channel is switched to change the tone limit.
 \section{Testing}
 We tested this by flipping different switches, pressing BTNU, and switching the audio channel.  We flipped multiple switches to see how drastically the wailing speed changes with the switches.
