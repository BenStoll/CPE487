\chapter{Lab 6: pong } \\
\index{Lab 6}
\index{introduction}
\label{Introduction}

\section{Lessons Learned}
    We learned how to create an interactive game using the Nexys board that would be displayed on a monitor through a VGA output.  We learned how to use a Pmod AD1 and potentiometer to control the bar for pong.

\label{Section::Lessons Learned}
    \section{Schematics and Block Diagrams}
        \includegraphics[width=100mm]{lab6/pong.png}
        \emph{Figure 6.1: Input and Output of the System}

\section{VHDL Architecture}
\begin{verbatim}
set_property -dict { PACKAGE_PIN E3 IOSTANDARD LVCMOS33 } [get_ports {clk_in}];
create_clock -add -name sys_clk_pin -period 10.00 -waveform {0 5} [get_ports {clk_in}];

set_property -dict { PACKAGE_PIN B11 IOSTANDARD LVCMOS33 } [get_ports { VGA_hsync }]; #IO_L4P_T0_15 Sch=vga_hs
set_property -dict { PACKAGE_PIN B12 IOSTANDARD LVCMOS33 } [get_ports { VGA_vsync }]; #IO_L3N_T0_DQS_AD1N_15 Sch=vga_vs

set_property -dict { PACKAGE_PIN B7 IOSTANDARD LVCMOS33 } [get_ports { VGA_blue[0] }]; #IO_L2P_T0_AD12P_35 Sch=vga_b[0]
set_property -dict { PACKAGE_PIN C7 IOSTANDARD LVCMOS33 } [get_ports { VGA_blue[1] }]; #IO_L4N_T0_35 Sch=vga_b[1]
set_property -dict { PACKAGE_PIN D7 IOSTANDARD LVCMOS33 } [get_ports { VGA_blue[2] }];
set_property -dict { PACKAGE_PIN D8 IOSTANDARD LVCMOS33 } [get_ports { VGA_blue[3] }];
set_property -dict { PACKAGE_PIN A3 IOSTANDARD LVCMOS33 } [get_ports { VGA_red[0] }]; #IO_L8N_T1_AD14N_35 Sch=vga_r[0]
set_property -dict { PACKAGE_PIN B4 IOSTANDARD LVCMOS33 } [get_ports { VGA_red[1] }]; #IO_L7N_T1_AD6N_35 Sch=vga_r[1]
set_property -dict { PACKAGE_PIN C5 IOSTANDARD LVCMOS33 } [get_ports { VGA_red[2] }]; #IO_L1N_T0_AD4N_35 Sch=vga_r[2]
set_property -dict { PACKAGE_PIN A4 IOSTANDARD LVCMOS33 } [get_ports { VGA_red[3] }];
set_property -dict { PACKAGE_PIN C6 IOSTANDARD LVCMOS33 } [get_ports { VGA_green[0] }]; #IO_L1P_T0_AD4P_35 Sch=vga_g[0]
set_property -dict { PACKAGE_PIN A5 IOSTANDARD LVCMOS33 } [get_ports { VGA_green[1] }]; #IO_L3N_T0_DQS_AD5N_35 Sch=vga_g[1]
set_property -dict { PACKAGE_PIN B6 IOSTANDARD LVCMOS33 } [get_ports { VGA_green[2] }]; #IO_L2N_T0_AD12N_35 Sch=vga_g[2]
set_property -dict { PACKAGE_PIN A6 IOSTANDARD LVCMOS33 } [get_ports { VGA_green[3] }];

set_property -dict { PACKAGE_PIN D18 IOSTANDARD LVCMOS33 } [get_ports { ADC_SDATA1 }]; #IO_L21N_T3_DQS_A18_15 Sch=ja[2]
set_property -dict { PACKAGE_PIN E18 IOSTANDARD LVCMOS33 } [get_ports { ADC_SDATA2 }]; #IO_L21P_T3_DQS_15 Sch=ja[3]
set_property -dict { PACKAGE_PIN G17 IOSTANDARD LVCMOS33 } [get_ports { ADC_SCLK }]; #IO_L18N_T2_A23_15 Sch=ja[4]
set_property -dict { PACKAGE_PIN C17 IOSTANDARD LVCMOS33 } [get_ports { ADC_CS }]; #IO_L20N_T3_A19_15 Sch=ja[1]

set_property -dict { PACKAGE_PIN N17 IOSTANDARD LVCMOS33 } [get_ports { btn0 }]; #IO_L9P_T1_DQS_14 Sch=btnc

set_property -dict { PACKAGE_PIN J15   IOSTANDARD LVCMOS33 } [get_ports { SW[0] }]; #IO_L24N_T3_RS0_15 Sch=sw[0]
set_property -dict { PACKAGE_PIN L16   IOSTANDARD LVCMOS33 } [get_ports { SW[1] }]; #IO_L3N_T0_DQS_EMCCLK_14 Sch=sw[1]
set_property -dict { PACKAGE_PIN M13   IOSTANDARD LVCMOS33 } [get_ports { SW[2] }]; #IO_L6N_T0_D08_VREF_14 Sch=sw[2]
set_property -dict { PACKAGE_PIN R15   IOSTANDARD LVCMOS33 } [get_ports { SW[3] }]; #IO_L13N_T2_MRCC_14 Sch=sw[3]
set_property -dict { PACKAGE_PIN R17   IOSTANDARD LVCMOS33 } [get_ports { SW[4] }]; #IO_L12N_T1_MRCC_14 Sch=sw[4]

set_property -dict {PACKAGE_PIN L18 IOSTANDARD LVCMOS33} [get_ports {SEG7_seg[0]}]
set_property -dict {PACKAGE_PIN T11 IOSTANDARD LVCMOS33} [get_ports {SEG7_seg[1]}]
set_property -dict {PACKAGE_PIN P15 IOSTANDARD LVCMOS33} [get_ports {SEG7_seg[2]}]
set_property -dict {PACKAGE_PIN K13 IOSTANDARD LVCMOS33} [get_ports {SEG7_seg[3]}]
set_property -dict {PACKAGE_PIN K16 IOSTANDARD LVCMOS33} [get_ports {SEG7_seg[4]}]
set_property -dict {PACKAGE_PIN R10 IOSTANDARD LVCMOS33} [get_ports {SEG7_seg[5]}]
set_property -dict {PACKAGE_PIN T10 IOSTANDARD LVCMOS33} [get_ports {SEG7_seg[6]}]

set_property -dict {PACKAGE_PIN U13 IOSTANDARD LVCMOS33} [get_ports {SEG7_anode[7]}]
set_property -dict {PACKAGE_PIN K2 IOSTANDARD LVCMOS33} [get_ports {SEG7_anode[6]}]
set_property -dict {PACKAGE_PIN T14 IOSTANDARD LVCMOS33} [get_ports {SEG7_anode[5]}]
set_property -dict {PACKAGE_PIN P14 IOSTANDARD LVCMOS33} [get_ports {SEG7_anode[4]}]
set_property -dict {PACKAGE_PIN J14 IOSTANDARD LVCMOS33} [get_ports {SEG7_anode[3]}]
set_property -dict {PACKAGE_PIN T9 IOSTANDARD LVCMOS33} [get_ports {SEG7_anode[2]}]
set_property -dict {PACKAGE_PIN J18 IOSTANDARD LVCMOS33} [get_ports {SEG7_anode[1]}]
set_property -dict {PACKAGE_PIN J17 IOSTANDARD LVCMOS33} [get_ports {SEG7_anode[0]}]
\end{verbatim}
\section{VHDL Models}
\begin{itemize}
    \item Data Flow
    \begin{verbatim}
        bat_n_ball.vhd
            input: serve input needed to serve ball (create ball and start it)
            output: draws bat and ball.  Causes ball to bounce with collisions. 
        adc_if.vhd
            input: serial data from both channels of ADC
            output: 12-bit parallel format data.  Drives bat_x of bat_n_ball module
        pong.vhd
            input: BTN0 button to initiate serve
            output: ckp prcoess generates timing signals for VGA and ADC modules.  
    \end{verbatim}
    \item Structural
        \begin {verbatim}
        Design Sources
            pong.vhd
                adc : adc_if.vhd
                add_bb : bat_n_ball.vhd
                vga_driver : vga_sync.vhd
                clk_wiz_0_inst : clk_wiz_0.vhd
                    U0 : clk_wiz_0_clk)wiz.vhd
                led1 : leddec16.vhd
        Contraints
            constrs_1
                pong.xdc
        \end{verbatim}
    \item Behavioral
        The program inputs come from the Pmod AD1 and potentiometer controlling the bat, the switches controlling the ball speed, and the BTN0 button starting a serve.
\end{itemize}
\section{VHDL Component Reuse}
    This section reuses the LED screen on the Nexys board to display score, the onboard buttons of the Nexys board to start a serve, the VGA output and monitor used for display, and the switches used to change the ball speed.
\section{VHDL Digital Circuits}
    \includegraphics[width=100mm]{lab6/pongimgthing.png}
    \emph{Figure 6.2: Image of the Digital Architecture and Circuit}
\section{State Machines}
    The system is displaying the bat and previous games score until a button is pressed.  When the BTN0 button is pressed, a game starts and the VGA output send out the visual signal for the game to continue, and there will be motion so long as the ball is in play, the PmodAD1 and potentiometer are being moved.
\section{testing}
    We tested this by interacting with all of the inputs.  We played pong on the monitor using the switches to change the ball speed, the BTN0 to initiate new serves, and the Pmod AD1 and potentiometer to move the bat.  We observed the monitor and LED display to ensure the controls were working correctly and to ensure the score was being kept.
