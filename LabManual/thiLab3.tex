\chapter{Lab 3: Bouncing Ball \\
\index{Lab 3}
\index{introduction}
\label{Introduction}}

\section{Lessons Learned \label{Section::Lessons Learned}}
    We learned how to use an FPGA to write to a display and how it changes with each cycle of the clock and then combined it with a previous lab (Adder Lab) to change the speed of the ball by flipping switches on the board and then we changed the color of the ball. After that, the final alteration that we made was to convert the moving ball to a pong style game with a paddle controlled by a dial to prevent the ball from touching the bottom of the screen. 

    We used the VGA port on the board to connect to a television. What this lab was teaching us was how to use different types of add ons to the board and how we would need to alter the code within. 
 \section{Schematics and Block Diagrams}
    The schematics
 \section{VHDL Architecture}
 
 \section{VHDL Models}
 
    \begin{enumerate}
        \item Data Flow Architecture
        \item Structural Modeling
        \item Behavioral Models
     \end{enumerate}

 \section{VHDL Component Reuse}
 \section{VHDL Digital Circuits}
 \section{State Machines}
 \section{Testing}
